\documentclass[12pt,a4paper]{article}
\usepackage[czech]{babel}
\usepackage[utf8]{inputenc}
\usepackage[T1]{fontenc}
\usepackage{lmodern}
\usepackage{graphicx}
\usepackage{multicol}
\usepackage{titling}
\usepackage{enumitem}
\textwidth 16cm \textheight 24.6cm
\topmargin -1.3cm
\oddsidemargin 0cm
\begin{document}
\pagestyle{empty}
\title{Úlohy na procvičení seznamů}
\date{\vspace{-10ex}}
\setlength{\droptitle}{-6em}
\maketitle

\setlength\parindent{0pt}

\begin{multicols}{2}

\section{Základní funkce}

Napište funkce nad seznamem čísel, které zjistí:

\begin{itemize}[noitemsep,nolistsep]
	\item součet všech čísel v seznamu,
	\item nejvyšší číslo v seznamu,
	\item zda se určitá hodnota vyskytuje v seznamu,
\end{itemize}

\begin{verbatim}
int MySum(List<int> numbers)
int MyMax(List<int> numbers)
bool MyIn(List<int> numbers,
          int number)
\end{verbatim}

\section{Součin}

Napište funkci, která vypočítá součin čísel v seznamu, ale ignoruje přitom
případné nuly.

\section{Zdvojení písmen}

Napište funkci, která vrátí nový řetězec, ve kterém bylo každé písmenko
zdvojeno.

\begin{verbatim}
string Duplication(stringtext)
Duplication("cs a") // ccss  aa
\end{verbatim}

\section{Cenzura}

Napište funkci, která zcenzuruje dodaný řetězec tak, že každé druhé písmeno nahradí za X.

\begin{verbatim}
string Censorship(string text)
Censorship("LosKarlos") // LXsXaXlXs
\end{verbatim}

\section{Počet A}

Napište funkci, která spočítá počet výskytů písmene (znaku) \texttt{A/a}.

\begin{verbatim}
int CountA(string text)
CountA("Liska Adelka") // 3
\end{verbatim}

\section{Znaky na stejných pozicích}

Napište funkci, která dostane dva řetězce a vypíše ty znaky, které jsou na
shodných pozicích stejné.

\begin{verbatim}
void StringIntersection(
  string left, string right)

StringIntersection("ZIRAFA", "KARAFA")
// R A F A
StringIntersection("PES", "KOCKA")
// (prazdny retezec)
\end{verbatim}

\section{Palindrom}

Napište funkci, která vrátí, zda je řetězec palindromem. Palindromem je takové
slovo či věta, která má při čtení v libovolném směru stejný význam, například
nepotopen či jelenovi pivo nelej (mezery můžete ignorovat).

\begin{verbatim}
void Palindrom(string text)
Palindrom("JELENOVIPIVONELEJ")
// true
\end{verbatim}

\section{Modifikace vs. vytváření seznamu}

Napište funkci \texttt{DoubleAll}, která dostane na vstupu seznam čísel a
každý jeho prvek vynásobí dvěma. Dále napište funkci \texttt{CreateDoubled},
která dostane na vstupu seznam čísel a vrátí nový seznam získaný ze vstupního
tak, že každý prvek vynásobí dvěma. Na rozdíl od předchozí funkce však nemění
předaný seznam.

\end{multicols}

%\noindent\rule{\textwidth}{1pt}
%\section*{Bonusové úlohy}

%\section{Zploštení}

%Napište funkci, jejímž vstupem je seznam seznamů a výstupem je seznam, který
%obsahuje prvky všech jednotlivých seznamů.
%\newpage

%\section*{Základy seznamů}

%\begin{verbatim}
%List<int> l = new List<int>();
%\end{verbatim}


\end{document}
