\documentclass[12pt,a4paper]{article}
\usepackage[czech]{babel}
\usepackage[utf8]{inputenc}
\usepackage[T1]{fontenc}
\usepackage{lmodern}
\usepackage{graphicx}
\usepackage{multicol}
\usepackage{titling}
\usepackage{enumitem}
\textwidth 16cm \textheight 24.6cm
\topmargin -1.3cm
\oddsidemargin 0cm
\begin{document}
\pagestyle{plain}
\title{Úlohy na procvičení práce se soubory}
\date{\vspace{-10ex}}
\setlength{\droptitle}{-6em}
\maketitle

\setlength\parindent{0pt}

\section{Frekvenční analýza}

Vytvořte program, který provede frekvenční analýzu textu v~zadaném souboru.
Frekvenční analýza vypíše, kolik se v~souboru nachází znaků od každého písmene.

Příklad:

\begin{verbatim}
a: 25
b: 5
c: 8
...
\end{verbatim}

\section{Frekvenční analýza podruhé}

Upravte předchozí program tak, aby výsledek zapisoval do zadaného souboru.

\section{Generování hesel}

Napište program, kterému zadáte jméno souboru a počet hesel a on do souboru
vygeneruje daný počet hesel. Hesla jsou náhodné kombinace znaků a číslic.

\begin{itemize}[noitemsep,nolistsep]
	\item Pokud chcete, můžete nechat uživatele zadat délku hesel.
	\item Pokud chcete, můžete hesla generovat také ze speciálních znaků.
\end{itemize}

\section{Průměrný počet slov ve větě}

Napište program, který analyzuje text v~zadaném souboru a vrátí průměrný počet
slov ve větě.

\section{CSV}

Napište program, který načte soubor \texttt{znamky.csv} a u~každého studenta
vypíše průměr všech jeho známek.

\newpage

\section*{Soubory v~C\#}

Soubor je lineární zápis dat na disku. My budeme pracovat hlavně s~textovými
soubory, kde je každý znak uložen jako nějaké konkrétní číslo. Soubory můžeme
číst a můžeme do nich zapisovat. V~C\# používáme pro práci se soubory objekty
\texttt{StreamReader} a \texttt{StreamWriter}.

\subsection*{Čtení ze souboru}

Čtení ze souboru může vypadat takto:

\begin{verbatim}
using (StreamReader sr = new StreamReader("muj_soubor.txt"))
{
  string prvni_radek = sr.ReadLine();
  // V promenne prvni_radek je obsah prvniho_radku souboru.
}
\end{verbatim}

Při práci se soubory nezapomeňte na začátek programu přidat
\texttt{using System.IO;}.

\subsection*{Zápis do souboru}

\begin{verbatim}
using (StreamWriter sw = new StreamWriter("muj_soubor.txt"))
{
  sw.WriteLine("Tento text se zapise na prvni radek.");
}
\end{verbatim}

Při otevření souboru pro zápis je soubor přepsán. Pokud soubor neexistuje, je
vytvořen jako prázdný.

Všimněte si, že práce se soubory probíhá velice podobně jako práce s~konzolí.
Také zde najdete funkce \texttt{ReadLine} a \texttt{WriteLine}.

\end{document}
