\documentclass[12pt,a4paper]{article}
\usepackage[unicode,colorlinks=true]{hyperref}
\usepackage[czech]{babel}
\usepackage[utf8]{inputenc}
\usepackage[T1]{fontenc}
\usepackage{lmodern}
\usepackage{graphicx}
\usepackage{multicol}
\usepackage{titling}
\textwidth 16cm \textheight 24.6cm
\topmargin -1.3cm
\oddsidemargin 0cm
\begin{document}
\pagestyle{empty}
\title{Úlohy na procvičení \texttt{for} cyklu}
\date{\vspace{-10ex}}
\setlength{\droptitle}{-6em}
\maketitle

\begin{multicols}{2}

\section{Hvězdičky}

Napište program, který na vstupu dostane číslo \texttt{n} a vypíše \texttt{n}
řádků s~hvězdičkami. Ukázka pro $n = 3$:
\begin{verbatim}
*
*
*
\end{verbatim}

\section{Více hvězdiček}

Upravte předchozí program tak, aby na každém řádku bylo 5 hvězdiček. Zkuste
použít 2 for cykly. Ukázka pro $n = 3$:
\begin{verbatim}
*****
*****
*****
\end{verbatim}

\section{Čtverec}

Napište program, který dostane na vstupu číslo \texttt{n} a vytvoří pomocí
textové grafiky čtverec o~rozměrech $n \times n$.\\
Ukázka tabulky pro $n = 5$:
\begin{verbatim}
# # # # #
# # # # #
# # # # #
# # # # #
# # # # #
\end{verbatim}

\section{Prázdný čtverec}

Upravte předchozí program tak, aby byl vnitřek čtverce tečkovaný.
\begin{verbatim}
# # # # #
# . . . #
# . . . #
# . . . #
# # # # #
\end{verbatim}

\section{Sudá čísla}
Napište program, který na vstupu dostane číslo \texttt{n}. Program vypíše
prvních \texttt{n} sudých čísel počínaje nulou.

\section{Dělitelé}
Napište program, který nechá uživatele zadat číslo \texttt{x}. Program vypíše
všechny dělitele čísla \texttt{x}.

\section{Násobilka}

Napište program, který dostane na vstupu číslo \texttt{n} a vypíše tabulku
násobilky pro čísla $1..n$.\\
Ukázka tabulky pro $n = 5$:

\begin{verbatim}
    1 2 3 4 5
    - - - - -
1 | 1 2 3 4 5
2 | 2 4 6 8 10
3 | 3 6 9 12 15
4 | 4 8 12 16 20
5 | 5 10 15 20 25
\end{verbatim}

\section{Maximum}

Napište program, který dostane na vstupu číslo \texttt{n} a vypíše tabulku
o~rozměrech $n \times n$, kde každá buňka obsahuje maximum z~čísla sloupce
a čísla řádku.\\
Ukázka tabulky pro $n = 5$:

\begin{verbatim}
    1 2 3 4 5
    - - - - -
1 | 1 2 3 4 5
2 | 2 2 3 4 5
3 | 3 3 3 4 5
4 | 4 4 4 4 5
5 | 5 5 5 5 5
\end{verbatim}

\end{multicols}

\newpage

\section*{Jak vlastně funguje \texttt{for} cyklus?}

For cyklus slouží k~opakování určité části programu.\\
Používá se, pokud v~programu víme, kolikrát chceme opakovat určitou část
programu.\\
Když máme počet opakování uložený v~proměnné \texttt{n}, můžeme psát:

\begin{verbatim}
for(int i = 0; i < n; i++)
{
    // kod k~opakovani
    // klidne vice radku
}
\end{verbatim}
Proměnná \texttt{i} postupně nabývá hodnot $0, 1, 2, 3, ..., n-1$.\\
Každému průchodu cyklem se říká iterace, máme třeba 1. iteraci, 2. iteraci atd.\\
Pomocí příkazů v~hlavičce cyklu nastavujeme, jakou hodnotu má mít proměnná \texttt{i}
před spuštěním cyklu, jaká je podmínka pro přechod do další iterace a co se
má dělat s~proměnnou \texttt{i} na konci každé iterace.

\subsection*{Příklad}

Mějme program:

\begin{verbatim}
for(int i = 0; i < 5; i++)
{
    Console.WriteLine(i);
}
\end{verbatim}
Tento program vypíše:

\begin{verbatim}
0
1
2
3
4
\end{verbatim}

\noindent\rule{\textwidth}{1pt}
\section*{Bonusové úlohy}

\begin{multicols}{2}

\subsection{Pyramida}

Napište program, který vypíše pyramidu o výšce, kterou zadá uživatel.

\begin{verbatim}
        #
      # # #
    # # # # #
  # # # # # # #
\end{verbatim}

\subsection{Posloupnost}

Napište program, který nechá uživatele zadat číslo \texttt{n}. Program vypíše
prvních \texttt{n} členů následující posloupnosti:

\noindent\texttt{1 1 2 1 2 3 1 2 3 4 1 2 3 4 5 ...}

\end{multicols}

\end{document}
