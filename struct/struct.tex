\documentclass[12pt,a4paper]{article}
\usepackage[czech]{babel}
\usepackage[utf8]{inputenc}
\usepackage[T1]{fontenc}
\usepackage{lmodern}
\usepackage{graphicx}
\usepackage{multicol}
\usepackage{titling}
\textwidth 16cm \textheight 24.6cm
\topmargin -1.3cm
\oddsidemargin 0cm
\begin{document}
\pagestyle{empty}
\title{Úlohy na procvičení struktur}
\date{\vspace{-12ex}}
\setlength{\droptitle}{-6em}
\maketitle

\section{Barevné kruhy}

Vytvořte strukturu \texttt{Circle} pro reprezentaci barevného kruhu na daných
souřadnicích.

\begin{verbatim}
struct Circle
{
  public Circle(int x, int y, uint radius, string color) { /* TODO */ }
}

Circle c = new Circle(10, 20, 5, "red");
Console.WriteLine(c.radius);
Console.WriteLine(c.color);
\end{verbatim}


\section{Obdélník}

Napište strukturu \texttt{Rectangle} pro reprezentaci obdélníku o dané šířce a
výšce a funkci pro výpočet obsahu předaného obdélníku.

\begin{verbatim}
struct Rectange { /* TODO */ }
static int Area(Rectangle r) { /* TODO */ }
\end{verbatim}


\section{Bod}

Napište strukturu pro reprezentaci bodu v rovině (parametry \texttt{x} a
\texttt{y}). Potom implementujte funkci, která vypočítá vzdálenost dvou bodů.

\begin{verbatim}
struct Point
{
  public Point(int x, int y) { /* TODO */ }
}

static double Distance(Point a, Point b) { /* TODO */ }
\end{verbatim}


\noindent\rule{\textwidth}{1pt}
\section*{Bonusové úlohy}

\section{Kniha}

Napište strukturu \texttt{Book} pro reprezentaci knihy (s atributy
\textit{název}, \textit{autor}, \textit{ISBN} a \textit{cena}).  Dále napište
funkce \texttt{print\_info} pro výpis informací o knize a \texttt{draw\_cover}
pro vykreslení obálky knihy.

\end{document}
