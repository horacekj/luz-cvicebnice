\documentclass[12pt,a4paper]{article}
\usepackage[czech]{babel}
\usepackage[utf8]{inputenc}
\usepackage[T1]{fontenc}
\usepackage{lmodern}
\usepackage{graphicx}
\usepackage{multicol}
\usepackage{titling}
\textwidth 16cm \textheight 24.6cm
\topmargin -1.3cm
\oddsidemargin 0cm
\begin{document}
\pagestyle{empty}
\title{Úlohy na procvičení tříd}
\date{\vspace{-12ex}}
\setlength{\droptitle}{-6em}
\setlength{\parindent}{0cm}
\maketitle

\section{Kruhy}

Napište třídu pro reprezentaci kruhu s daným středem a poloměrem. Implementujte
metody pro výpočet obvodu, obsahu a vrácení informačního řetězce.

\begin{verbatim}
class Circle
{
  public int GetPerimeter() { /* vrati obvod kruhu */ }
  public int GetArea() { /* vrati obsah */ }
  public int Info() { /* vrati informacni string */ }
}

Circle c = new Circle(new Point(-130, -130), 100);
Console.WriteLine(c.GetPerimeter()) // Circle at (-130, -130) with radius 100
Console.WriteLine(c.GetArea()) // 31415.926535897932
\end{verbatim}

\section{Zásobník}

Vytvořte třídu pro reprezentaci zásobníku podporující operace přidání prvku,
odebrání prvku a test na prázdnost.

\begin{verbatim}
class Stack
{
  public void Push(int val) { /* TODO */ }
  public int Pop() { /* TODO */ }
  public bool IsEmpty() { /* TODO */ }
}
\end{verbatim}

\section{Knihovna}

Vytvořte třídu \texttt{Library} pro reprezentaci kolekce knih s metodami pro
\begin{enumerate}
	\setlength\itemsep{0em}
	\setlength{\parskip}{0pt}
	\item přidání knihy,
	\item odebrání knihy,
	\item nalezení knihy podle názvu nebo ISBN,
	\item nalezení všech knih daného autora,
	\item nalezení všech knih s cenu pod zadanou mez.
\end{enumerate}

Pro reprezentaci knih využijte třídu \texttt{Book} ze cvičení Kniha.

% Metodické poznámky:
% 1. ukázat metody
% 2. ukázat private, public
% 3. ukázat strukturu Point
% 4. ukázat vytváření privátních atributů (a konvence pro pojmenování)

\end{document}
