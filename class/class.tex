\documentclass[12pt,a4paper]{article}
\usepackage[czech]{babel}
\usepackage[utf8]{inputenc}
\usepackage[T1]{fontenc}
\usepackage{lmodern}
\usepackage{graphicx}
\usepackage{multicol}
\usepackage{titling}
\textwidth 16cm \textheight 24.6cm
\topmargin -1.3cm
\oddsidemargin 0cm
\begin{document}
\pagestyle{empty}
\title{Úlohy na procvičení tříd}
\date{\vspace{-12ex}}
\setlength{\droptitle}{-6em}
\setlength{\parindent}{0cm}
\maketitle

\section{Kruhy}

Napište třídu pro reprezentaci kruhu s daným středem a poloměrem. Implementujte
metody pro výpočet obvodu, obsahu a vrácení informačního řetězce.

\begin{verbatim}
class Circle
{
  public double GetPerimeter() { /* vrati obvod kruhu */ }
  public double GetArea() { /* vrati obsah */ }
  public string Info() { /* vrati informacni string */ }
}

Circle c = new Circle(new Point(-130, -130), 100);
Console.WriteLine(c.GetPerimeter()) // 628.3
Console.WriteLine(c.Info()) // Circle at (-130, -130) with radius 100
Console.WriteLine(c.GetArea()) // 31415.926535897932
\end{verbatim}

Nezapomeňte do třídy \texttt{Circle} dopsat konstruktor. Pro využití struktury
\texttt{Point} napište na začátek programu: \texttt{using System.Drawing;}
a proveďte přidání knihovny \textit{System.Drawing} (Project $\rightarrow$
Add Reference $\rightarrow$ System.Drawing $\rightarrow$ Select).

\section{Zásobník}

Vytvořte třídu pro reprezentaci zásobníku podporující operace přidání prvku,
odebrání prvku a test na prázdnost.

\begin{verbatim}
class Stack
{
  public void Push(int val) { /* TODO */ } // pridani prvku
  public int Pop() { /* TODO */ } // odebrani prvku
  public bool IsEmpty() { /* TODO */ } // true kdyz je zasobnik prazdny
}
\end{verbatim}

Nezapomeňte na začátek programu napsat
\texttt{using System.Collections.Generic;}.

\section{Knihovna}

Vytvořte třídu \texttt{Library} pro reprezentaci kolekce knih s metodami pro
\begin{enumerate}
	\setlength\itemsep{0em}
	\setlength{\parskip}{0pt}
	\item přidání knihy,
	\item odebrání knihy,
	\item nalezení knihy podle názvu nebo ISBN,
	\item nalezení všech knih daného autora,
	\item nalezení všech knih s cenu pod zadanou mez.
\end{enumerate}

Pro reprezentaci knih využijte třídu \texttt{Book} ze cvičení Kniha.

% Metodické poznámky:
% 1. ukázat metody
% 2. ukázat private, public
% 3. ukázat strukturu Point
% 4. ukázat vytváření privátních atributů (a konvence pro pojmenování)

\newpage

\section*{Jak fungují třídy?}

Třída je předpis pro vytváření instancí. Oproti strukturám je větší, používá
se pro popis složitějších objektů (např. člověk, počítač, bagr, ...). Třída
typicky obsahuje metody.

Metoda je funkce, kterou můžeme volat na třídě.

Příklad:

\begin{verbatim}
class Rectangle
{
  int w;
  int h;

  public int GetArea() { return w * h; }
}
\end{verbatim}

\texttt{GetArea} je metoda třídy \texttt{Rectangle}, která vrací obsah čtverce.

Můžeme jí zavolat:

\begin{verbatim}
Recangle r = new Rectangle();
r.w = 10;
r.h = 10;
Console.WriteLine(r.GetArea()); // vypise 100
\end{verbatim}

Metoda se volá vždy na instanci. Metoda má přístup k hodnotám atributů. Metoda
může atributy číst i měnit.

Metoda může být privátní (\texttt{private}) nebo veřejná (\texttt{public}).
Privátní metody lze volat jen z metod třídy, veřejné lze volata z vnější.
Kdyby v předchozím příkladu bylo \texttt{private int GetArea}, nelze metodu
\texttt{GetArea} zavolat a funkce \texttt{Main}.

Privátní atributy pojmenovávejte tak, aby jejich jména začínala podtržítkem!

V konstruktoru se typicky píše inicializace složitějších atributů. Příklad:

\begin{verbatim}
class NakupniSeznam
{
  private List<string> _l; // tady private nemusi byt

  public NakupniSeznam() // konstruktor
  {
    this._l = new List<string>(); // tady this nemusi byt
  }

  public void Add(string name) { _l.Add(name); }
  public void Remove(string name) { _l.Remove(name); }
}
\end{verbatim}

V tomto příkladu se s privátním seznamem \texttt{\_l} pracuje jen pomocí metod
\texttt{Add} a \texttt{Remove}.

\end{document}
